%
%  abstract.tex  2013-07-01  Derrick Kearney
%
%  details regarding why hubcheck is novel
%

\documentclass[letterpaper]{article}
\usepackage{times}
\usepackage{helvet}
\usepackage{courier}
\usepackage{graphicx}
\usepackage{enumerate}
\usepackage{amssymb}
\usepackage{amsmath}
\DeclareMathOperator*{\argmax}{argmax}

\begin{document}


\section{Problem being addressed}

\textbf{
State the problem the contribution helps solve. A good problem is one that is
very important; a non-specialist can understand the importance; and the problem
can be stated in a single sentence.
}

The hub is a website made up of components. After the website's software is
updated, testing of the components is performed by hand. Testing each component
by hand is time consuming and error prone.  It promotes variations in testing
where each iteration of testing can be performed a different way or not
performed if forgotten. It also promotes spot checking of fixes to regressions
instead of retesting everything because the time commitment to test everything
again is so high.

HUBcheck provides libraries which can be used to help automate the testing of
users interact with hub components that would otherwise be performed by hand.
When compared to testing by hand, HUBcheck reduces testing time, increases test
coverage, and provides a reliable way to reproduce errors.

\section{Metric to measure the size of the problem}

\textbf{
Say how one can measure the size of the problem. If a problem is quantifiable
by a clear metric, it will be easier to write the paper and get it accepted.
}

%1) How long does it take to test a website by hand?
%3 people testing, about 2 weeks
%2) What components are covered? To what depth are components covered
%groups, resources, 
%3) fixes to regressions are spot tested, instead of retesting everything.

The current method of testing takes on the order of weeks to perform. Test
cases are not well documented, so the depth of coverage of the hub components
is unknown. Additionally, the testing environment is constantly being updated
with new software, which invalidates the test results as testing occurs.

Things to measure:

\begin{enumerate}
\item Time needed to test hub in days or weeks
\item Number of components covered
\item Number of tests performed
\item Number of regressions found
\end{enumerate}


\section{State of the art solutions to the problem}

\textbf{
Say to what extent related work has solved the problem. Mention the
directly-related work that you'd like to get better than - the best
alternative(s). Give quantitative answers, using the metric in \#2. If the
best alternative is your own work, state this also. A paper get more easily
accepted if the best alternative is not your own; papers for which the best
alternative of all contributions is your own work have a harder time getting
accepted. if not obvious from what you say, state the gap between the state
of the art and the overall goal expressed in \#1 explicitly.
}

The current way to test how a user interacts with a hub is to generate test
users, login to the hub as the test user and perform actions. This process is
tedious because of the detail related to properly maintaining user accounts and
consistently running tests. Testing a new release takes three people about two
weeks. During the testing process many of the hub's components are tested,
including Dashboard, Feedback, Jobs, Knowledge Base, Questions and Answers,
Wishlists, and Tools.

% http://hubzero.org/projects/testingproject/files/?action=download&subdir=QA+Testing%2FEmily%2FDashboard&file=DashboardIE-Registered-Emily.docx
%Dashboard

% http://hubzero.org/projects/testingproject/files/?action=download&subdir=QA+Testing%2FEmily%2FFeedback&file=FeedbackIE-Emily-RegisteredUser.docx
%Feedback

% http://hubzero.org/projects/testingproject/files/?action=download&subdir=QA+Testing%2FEmily%2FJobs+and+Resume&file=Jobs-Resume-IE-Registered-Emily.docx
%Jobs

% http://hubzero.org/projects/testingproject/files/?action=download&subdir=QA+Testing%2FEmily%2FKnowledge+Base&file=KnowledgeBaseIE-Registered-Emily.docx
%Knowledge Base

% http://hubzero.org/projects/testingproject/files/?action=download&subdir=QA+Testing%2FEmily%2FQA&file=QuestionsAnswersIE-Registered-Emily.docx
%Questions and Answers

% http://hubzero.org/projects/testingproject/files/?action=download&subdir=QA+Testing%2FEmily%2FWishlist&file=WishlistIE-registered-Emily.docx
%Wishlists



\section{Key idea to move beyond the state of the art}

\textbf{
% State the key idea underlying your contribution. This key idea is what
% enabled your new technique to go beyond the state of the art. It should be
% something non-obvious; realizing this key idea is what makes your technique
% novel. The key idea should be readable in plain English ( in the
% introduction). Avoid complex formula that would take the reader time to
% understand. Mathematical expressions should go in the sections that describe
% the realizations of the ideas.
}

The HUBcheck libraries help users automate tasks being performed through a
secure shell (SSH) and through the web browser. Building on top of Tcl's Expect
and Python's Paramiko libraries, HUBcheck provides functions to easily automate
the access of different environments through SSH. Similarly, HUBcheck uses the
Selenium library to automate user interactions with the hub website. HUBcheck
libraries provide an abstraction of hub components, which can be used to write
maintainable test cases.



\section{Expected gain}

\textbf{
How much narrower do you expect the gap between state of the art and ultimate
goal (\#1) to become, owing to your contribution? if you have finished your
experiments, you may replace the estimate with the measured results. As many
papers do or should start out before measurements have been collected, it
also helps set the readers' expectations.
}

By using HUBcheck to write test cases for hub components, a hub can be
validated in under a day. The decrease in testing time will encourage the
adoption of automated testing into the development cycle and the quick
identification and resolution of bugs and regressions that would otherwise make
it to production environments.

\section{State how you will demonstrate progress}

\textbf{
What are the experiments you will do (or have done) to demonstrate progress?
Make sure the experiments result in a figure that uses the metric Column B on
its y-axis. Often, such figures have a number of benchmarks on the x-axis and
show data points for both the state of the art technique and your new
technique. Additional results may show breakdowns of the overall
"performance" into interesting components. Be sure the measure each
contribution separately, in addition to a possible combined progress result.
}

Five case studies demonstrate the effectiveness of using the HUBcheck libraries
to help automate tasks related to the hub. The case studies review the use of
HUBcheck libraries to:

\begin{enumerate}
\item Validate of tool session containers
\item Validate of website components
\item Identify bad links on the hub
\item Perform nightly builds and test for the Rappture Toolkit
\item Automate hub services like nanohubu and contribtool
\end{enumerate}

\end{document}

