%
%  mydefs.tex  2007-03-19  Mark Senn  http://www.ecn.purdue.edu/~mark
%
%  Command definitions that can be used in all documents that have
%      %
%  mydefs.tex  2007-03-19  Mark Senn  http://www.ecn.purdue.edu/~mark
%
%  Command definitions that can be used in all documents that have
%      %
%  mydefs.tex  2007-03-19  Mark Senn  http://www.ecn.purdue.edu/~mark
%
%  Command definitions that can be used in all documents that have
%      %
%  mydefs.tex  2007-03-19  Mark Senn  http://www.ecn.purdue.edu/~mark
%
%  Command definitions that can be used in all documents that have
%      \input{mydefs}
%

%%%%%%%%%%%%%%%%%%%%%%%%%%%%%%%%%%%%%%%%%%%%%%%%%%
% setup code single spacing caption for all floats

\captionsetup{
  font={stretch=1},
}

%%%%%%%%%%%%%%%%%%%%%%%%%%%%%%%%%%%%%%%%%%%%%%%%%%


%%%%%%%%%%%%%%%%%%%%%%%%%%%%%%%%%%%%%%%%%%%%%%%%%%
% setup special fonts/format (xformat,xf) for talking about:
%   module
%   classes
%   class methods
%   class attributes (data members)
%   objects
%   locator
%   locator family
%   inline code
%   html link name
%   parameter name
%   uri scheme (http, https, mailto, webdav)
%   program name

\newcommand{\xfmodule}[1]{\texttt{#1}}
\newcommand{\xfclass}[1]{\texttt{#1}}
\newcommand{\xfmethod}[1]{\texttt{#1}}
\newcommand{\xfattribute}[1]{\texttt{#1}}
\newcommand{\xfobject}[1]{\texttt{#1}}
\newcommand{\xflocator}[1]{\texttt{#1}}
\newcommand{\xflocfam}[1]{\texttt{#1}}
\newcommand{\xfinlinecode}[1]{\texttt{#1}}
\newcommand{\xfhtmllink}[1]{\texttt{#1}}
\newcommand{\xfparameter}[1]{\texttt{#1}}
\newcommand{\xfurischeme}[1]{\texttt{#1}}
\newcommand{\xfprogramname}[1]{\textbf{#1}}


%%%%%%%%%%%%%%%%%%%%%%%%%%%%%%%%%%%%%%%%%%%%%%%%%%


%%%%%%%%%%%%%%%%%%%%%%%%%%%%%%%%%%%%%%%%%%%%%%%%%%
% setup code listing captions
%
% i could never figure out why the colorbox seems so indented
% so we just multiply our fboxsep by 7 and that accounts for
% the indentation. had to adjust framexleftmargin and
% framexrightmargin to match.
%

\DeclareCaptionFont{white}{\color{white}}
\DeclareCaptionFormat{listing}{
  \colorbox[cmyk]{0.43, 0.35, 0.35, 0.01}{
    \parbox{\dimexpr\textwidth-7\fboxsep\relax} {
      #1#2#3
    }
  }
}

\captionsetup[lstlisting]{
  format=listing,
  labelfont=white,
  textfont=white,
  singlelinecheck=false,
  margin=0pt,
}

%%%%%%%%%%%%%%%%%%%%%%%%%%%%%%%%%%%%%%%%%%%%%%%%%%


%%%%%%%%%%%%%%%%%%%%%%%%%%%%%%%%%%%%%%%%%%%%%%%%%%
% setting for code listings

\lstset{ %
  basicstyle=\scriptsize\ttfamily,    % the size of the fonts that are used for the code
  breaklines=true,                    % sets automatic line breaking
  captionpos=t,                       % sets the caption-position to bottom
  belowcaptionskip=15px,              % add some space under the caption and above the code
  frame=lines,                        % adds a frame around the code
  numbers=left,                       % where to put the line-numbers; possible values are (none, left, right)
  numbersep=5pt,                      % how far the line-numbers are from the code
  numberstyle=\tiny\color{black},     % the style that is used for the line-numbers
  showspaces=false,                   % show spaces everywhere adding particular underscores;
  showstringspaces=false,             % underline spaces within strings only
  showtabs=false,                     % show tabs within strings adding particular underscores
  stepnumber=1,                       % the step between two line-numbers. If it's 1, each line will be numbered
  stringstyle=\color{black}\ttfamily\textbf, % string literal style
  tabsize=2,                          % sets default tabsize to 2 spaces
  xleftmargin=5ex,                    % space between page margin and line numbers/code/frames
  framexleftmargin=13pt,               % left width of tob/bottom frame, starts at code, excludes line numbers
  framexrightmargin=-3pt,              % right width of top/bottom frame, starts at textwidth?
%  framexbottommargin=5pt,
  extendedchars=true,
  belowskip=15pt,
}

% load languages we will be using
\lstloadlanguages{
  XML,
  HTML,
  Python,
}

\lstnewenvironment{xcode}[1]
  {
    \noindent
    \minipage{\linewidth}
%    \lstset{tabsize=4, belowcaptionskip=1\baselineskip, #1}
    \lstset{tabsize=4, #1}
  }%
  {
    \endminipage
  }

%%%%%%%%%%%%%%%%%%%%%%%%%%%%%%%%%%%%%%%%%%%%%%%%%%

%

%%%%%%%%%%%%%%%%%%%%%%%%%%%%%%%%%%%%%%%%%%%%%%%%%%
% setup code single spacing caption for all floats

\captionsetup{
  font={stretch=1},
}

%%%%%%%%%%%%%%%%%%%%%%%%%%%%%%%%%%%%%%%%%%%%%%%%%%


%%%%%%%%%%%%%%%%%%%%%%%%%%%%%%%%%%%%%%%%%%%%%%%%%%
% setup special fonts/format (xformat,xf) for talking about:
%   module
%   classes
%   class methods
%   class attributes (data members)
%   objects
%   locator
%   locator family
%   inline code
%   html link name
%   parameter name
%   uri scheme (http, https, mailto, webdav)
%   program name

\newcommand{\xfmodule}[1]{\texttt{#1}}
\newcommand{\xfclass}[1]{\texttt{#1}}
\newcommand{\xfmethod}[1]{\texttt{#1}}
\newcommand{\xfattribute}[1]{\texttt{#1}}
\newcommand{\xfobject}[1]{\texttt{#1}}
\newcommand{\xflocator}[1]{\texttt{#1}}
\newcommand{\xflocfam}[1]{\texttt{#1}}
\newcommand{\xfinlinecode}[1]{\texttt{#1}}
\newcommand{\xfhtmllink}[1]{\texttt{#1}}
\newcommand{\xfparameter}[1]{\texttt{#1}}
\newcommand{\xfurischeme}[1]{\texttt{#1}}
\newcommand{\xfprogramname}[1]{\textbf{#1}}


%%%%%%%%%%%%%%%%%%%%%%%%%%%%%%%%%%%%%%%%%%%%%%%%%%


%%%%%%%%%%%%%%%%%%%%%%%%%%%%%%%%%%%%%%%%%%%%%%%%%%
% setup code listing captions
%
% i could never figure out why the colorbox seems so indented
% so we just multiply our fboxsep by 7 and that accounts for
% the indentation. had to adjust framexleftmargin and
% framexrightmargin to match.
%

\DeclareCaptionFont{white}{\color{white}}
\DeclareCaptionFormat{listing}{
  \colorbox[cmyk]{0.43, 0.35, 0.35, 0.01}{
    \parbox{\dimexpr\textwidth-7\fboxsep\relax} {
      #1#2#3
    }
  }
}

\captionsetup[lstlisting]{
  format=listing,
  labelfont=white,
  textfont=white,
  singlelinecheck=false,
  margin=0pt,
}

%%%%%%%%%%%%%%%%%%%%%%%%%%%%%%%%%%%%%%%%%%%%%%%%%%


%%%%%%%%%%%%%%%%%%%%%%%%%%%%%%%%%%%%%%%%%%%%%%%%%%
% setting for code listings

\lstset{ %
  basicstyle=\scriptsize\ttfamily,    % the size of the fonts that are used for the code
  breaklines=true,                    % sets automatic line breaking
  captionpos=t,                       % sets the caption-position to bottom
  belowcaptionskip=15px,              % add some space under the caption and above the code
  frame=lines,                        % adds a frame around the code
  numbers=left,                       % where to put the line-numbers; possible values are (none, left, right)
  numbersep=5pt,                      % how far the line-numbers are from the code
  numberstyle=\tiny\color{black},     % the style that is used for the line-numbers
  showspaces=false,                   % show spaces everywhere adding particular underscores;
  showstringspaces=false,             % underline spaces within strings only
  showtabs=false,                     % show tabs within strings adding particular underscores
  stepnumber=1,                       % the step between two line-numbers. If it's 1, each line will be numbered
  stringstyle=\color{black}\ttfamily\textbf, % string literal style
  tabsize=2,                          % sets default tabsize to 2 spaces
  xleftmargin=5ex,                    % space between page margin and line numbers/code/frames
  framexleftmargin=13pt,               % left width of tob/bottom frame, starts at code, excludes line numbers
  framexrightmargin=-3pt,              % right width of top/bottom frame, starts at textwidth?
%  framexbottommargin=5pt,
  extendedchars=true,
  belowskip=15pt,
}

% load languages we will be using
\lstloadlanguages{
  XML,
  HTML,
  Python,
}

\lstnewenvironment{xcode}[1]
  {
    \noindent
    \minipage{\linewidth}
%    \lstset{tabsize=4, belowcaptionskip=1\baselineskip, #1}
    \lstset{tabsize=4, #1}
  }%
  {
    \endminipage
  }

%%%%%%%%%%%%%%%%%%%%%%%%%%%%%%%%%%%%%%%%%%%%%%%%%%

%

%%%%%%%%%%%%%%%%%%%%%%%%%%%%%%%%%%%%%%%%%%%%%%%%%%
% setup code single spacing caption for all floats

\captionsetup{
  font={stretch=1},
}

%%%%%%%%%%%%%%%%%%%%%%%%%%%%%%%%%%%%%%%%%%%%%%%%%%


%%%%%%%%%%%%%%%%%%%%%%%%%%%%%%%%%%%%%%%%%%%%%%%%%%
% setup special fonts/format (xformat,xf) for talking about:
%   module
%   classes
%   class methods
%   class attributes (data members)
%   objects
%   locator
%   locator family
%   inline code
%   html link name
%   parameter name
%   uri scheme (http, https, mailto, webdav)
%   program name

\newcommand{\xfmodule}[1]{\texttt{#1}}
\newcommand{\xfclass}[1]{\texttt{#1}}
\newcommand{\xfmethod}[1]{\texttt{#1}}
\newcommand{\xfattribute}[1]{\texttt{#1}}
\newcommand{\xfobject}[1]{\texttt{#1}}
\newcommand{\xflocator}[1]{\texttt{#1}}
\newcommand{\xflocfam}[1]{\texttt{#1}}
\newcommand{\xfinlinecode}[1]{\texttt{#1}}
\newcommand{\xfhtmllink}[1]{\texttt{#1}}
\newcommand{\xfparameter}[1]{\texttt{#1}}
\newcommand{\xfurischeme}[1]{\texttt{#1}}
\newcommand{\xfprogramname}[1]{\textbf{#1}}


%%%%%%%%%%%%%%%%%%%%%%%%%%%%%%%%%%%%%%%%%%%%%%%%%%


%%%%%%%%%%%%%%%%%%%%%%%%%%%%%%%%%%%%%%%%%%%%%%%%%%
% setup code listing captions
%
% i could never figure out why the colorbox seems so indented
% so we just multiply our fboxsep by 7 and that accounts for
% the indentation. had to adjust framexleftmargin and
% framexrightmargin to match.
%

\DeclareCaptionFont{white}{\color{white}}
\DeclareCaptionFormat{listing}{
  \colorbox[cmyk]{0.43, 0.35, 0.35, 0.01}{
    \parbox{\dimexpr\textwidth-7\fboxsep\relax} {
      #1#2#3
    }
  }
}

\captionsetup[lstlisting]{
  format=listing,
  labelfont=white,
  textfont=white,
  singlelinecheck=false,
  margin=0pt,
}

%%%%%%%%%%%%%%%%%%%%%%%%%%%%%%%%%%%%%%%%%%%%%%%%%%


%%%%%%%%%%%%%%%%%%%%%%%%%%%%%%%%%%%%%%%%%%%%%%%%%%
% setting for code listings

\lstset{ %
  basicstyle=\scriptsize\ttfamily,    % the size of the fonts that are used for the code
  breaklines=true,                    % sets automatic line breaking
  captionpos=t,                       % sets the caption-position to bottom
  belowcaptionskip=15px,              % add some space under the caption and above the code
  frame=lines,                        % adds a frame around the code
  numbers=left,                       % where to put the line-numbers; possible values are (none, left, right)
  numbersep=5pt,                      % how far the line-numbers are from the code
  numberstyle=\tiny\color{black},     % the style that is used for the line-numbers
  showspaces=false,                   % show spaces everywhere adding particular underscores;
  showstringspaces=false,             % underline spaces within strings only
  showtabs=false,                     % show tabs within strings adding particular underscores
  stepnumber=1,                       % the step between two line-numbers. If it's 1, each line will be numbered
  stringstyle=\color{black}\ttfamily\textbf, % string literal style
  tabsize=2,                          % sets default tabsize to 2 spaces
  xleftmargin=5ex,                    % space between page margin and line numbers/code/frames
  framexleftmargin=13pt,               % left width of tob/bottom frame, starts at code, excludes line numbers
  framexrightmargin=-3pt,              % right width of top/bottom frame, starts at textwidth?
%  framexbottommargin=5pt,
  extendedchars=true,
  belowskip=15pt,
}

% load languages we will be using
\lstloadlanguages{
  XML,
  HTML,
  Python,
}

\lstnewenvironment{xcode}[1]
  {
    \noindent
    \minipage{\linewidth}
%    \lstset{tabsize=4, belowcaptionskip=1\baselineskip, #1}
    \lstset{tabsize=4, #1}
  }%
  {
    \endminipage
  }

%%%%%%%%%%%%%%%%%%%%%%%%%%%%%%%%%%%%%%%%%%%%%%%%%%

%

%%%%%%%%%%%%%%%%%%%%%%%%%%%%%%%%%%%%%%%%%%%%%%%%%%
% setup code single spacing caption for all floats

\captionsetup{
  font={stretch=1},
}

%%%%%%%%%%%%%%%%%%%%%%%%%%%%%%%%%%%%%%%%%%%%%%%%%%


%%%%%%%%%%%%%%%%%%%%%%%%%%%%%%%%%%%%%%%%%%%%%%%%%%
% setup special fonts/format (xformat,xf) for talking about:
%   module
%   classes
%   class methods
%   class attributes (data members)
%   objects
%   locator
%   locator family
%   inline code
%   html link name
%   parameter name
%   uri scheme (http, https, mailto, webdav)
%   program name

\newcommand{\xfmodule}[1]{\texttt{#1}}
\newcommand{\xfclass}[1]{\texttt{#1}}
\newcommand{\xfmethod}[1]{\texttt{#1}}
\newcommand{\xfattribute}[1]{\texttt{#1}}
\newcommand{\xfobject}[1]{\texttt{#1}}
\newcommand{\xflocator}[1]{\texttt{#1}}
\newcommand{\xflocfam}[1]{\texttt{#1}}
\newcommand{\xfinlinecode}[1]{\texttt{#1}}
\newcommand{\xfhtmllink}[1]{\texttt{#1}}
\newcommand{\xfparameter}[1]{\texttt{#1}}
\newcommand{\xfurischeme}[1]{\texttt{#1}}
\newcommand{\xfprogramname}[1]{\textbf{#1}}


%%%%%%%%%%%%%%%%%%%%%%%%%%%%%%%%%%%%%%%%%%%%%%%%%%


%%%%%%%%%%%%%%%%%%%%%%%%%%%%%%%%%%%%%%%%%%%%%%%%%%
% setup code listing captions
%
% i could never figure out why the colorbox seems so indented
% so we just multiply our fboxsep by 7 and that accounts for
% the indentation. had to adjust framexleftmargin and
% framexrightmargin to match.
%

\DeclareCaptionFont{white}{\color{white}}
\DeclareCaptionFormat{listing}{
  \colorbox[cmyk]{0.43, 0.35, 0.35, 0.01}{
    \parbox{\dimexpr\textwidth-7\fboxsep\relax} {
      #1#2#3
    }
  }
}

\captionsetup[lstlisting]{
  format=listing,
  labelfont=white,
  textfont=white,
  singlelinecheck=false,
  margin=0pt,
}

%%%%%%%%%%%%%%%%%%%%%%%%%%%%%%%%%%%%%%%%%%%%%%%%%%


%%%%%%%%%%%%%%%%%%%%%%%%%%%%%%%%%%%%%%%%%%%%%%%%%%
% setting for code listings

\lstset{ %
  basicstyle=\scriptsize\ttfamily,    % the size of the fonts that are used for the code
  breaklines=true,                    % sets automatic line breaking
  captionpos=t,                       % sets the caption-position to bottom
  belowcaptionskip=15px,              % add some space under the caption and above the code
  frame=lines,                        % adds a frame around the code
  numbers=left,                       % where to put the line-numbers; possible values are (none, left, right)
  numbersep=5pt,                      % how far the line-numbers are from the code
  numberstyle=\tiny\color{black},     % the style that is used for the line-numbers
  showspaces=false,                   % show spaces everywhere adding particular underscores;
  showstringspaces=false,             % underline spaces within strings only
  showtabs=false,                     % show tabs within strings adding particular underscores
  stepnumber=1,                       % the step between two line-numbers. If it's 1, each line will be numbered
  stringstyle=\color{black}\ttfamily\textbf, % string literal style
  tabsize=2,                          % sets default tabsize to 2 spaces
  xleftmargin=5ex,                    % space between page margin and line numbers/code/frames
  framexleftmargin=13pt,               % left width of tob/bottom frame, starts at code, excludes line numbers
  framexrightmargin=-3pt,              % right width of top/bottom frame, starts at textwidth?
%  framexbottommargin=5pt,
  extendedchars=true,
  belowskip=15pt,
}

% load languages we will be using
\lstloadlanguages{
  XML,
  HTML,
  Python,
}

\lstnewenvironment{xcode}[1]
  {
    \noindent
    \minipage{\linewidth}
%    \lstset{tabsize=4, belowcaptionskip=1\baselineskip, #1}
    \lstset{tabsize=4, #1}
  }%
  {
    \endminipage
  }

%%%%%%%%%%%%%%%%%%%%%%%%%%%%%%%%%%%%%%%%%%%%%%%%%%
