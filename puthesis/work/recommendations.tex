%
%  recommendations.tex  2007-02-06  Mark Senn  http://www.ecn.purdue.edu/~mark
%

% incorporate hubcheck into the development process. at the very least, when we
% roll out a new feature or component on the hub, we should be asking "what is
% the test plan". it also helps to have light documentation or requirements
% describing how the component is supposed to work. these requirements are the
% basis for hubcheck tests.
%
% selenium is not the only website checking software out there. other web
% automation libraries can be used instead. the work done in hubcheck can be
% used as a proof of concept and can possibly be applied to other software.
%
% with respect to building more page objects and testing more web components,
% the hubzero team should probably make it a priority to focus on building
% tests for one component at a time since resources seem to be streached and
% the team has already published components that don't have an automated
% testing plan.
%
% it would also help if the team had test environments where developers would
% quickly build a clean version of the hub, running the recommended setup or a
% clone of one of our existing hubs. the developer could then load their new
% software on this test system and run hubcheck against the test system. the
% hubcheck tests should include tests for their new feature so they can make
% sure their code works, as well as tests from previously accepted components
% to make sure the new feature works with the currently stable code.
%
% recommendations and future work for hubcheck.
%
% incorporate googletest or some other framework for testing Rappture's C/C++
% code.
%
% currently hubcheck makes it very easy to use pre-installed page objects and
% very hard to use page objects that are not a part of the standard
% installation. this limits the user's ability to build and use page objects
% for web pages other than the hub when using hubcheck. hubcheck libraries can
% be used for websites that are not hubs. it should be easier for users to
% build page objects that they store locally, outside of the standard
% installation, and use with their own scripts.
%
% like to separate the hubcheck tests and hub page objects from the hubcheck
% library.
%
% finish the hubcheck.bashshell module which builds upon the pexpect module, a
% python implementation of Tcl's Expect. the hubcheck.bashshell module aims to
% implement an interface similar to that of the hubcheck.sshshell module,
% providing the usual send() and expect() methods, but also provides an
% execute() method. the execute() method can run small scripts in the form of a
% list of commands, checking the exit status of each command, and optionally
% raising an exception if a command fails. this method reduces the usual cycle
% of sending a command, expecting a result, sending another command to check
% the exit status, and expecting the exit status, down to a single method call
% with the command or list of commands that should be run.


% future work
% 1. hubcheck library improvements.
%   0. **creating test accounts and storing passwords**
%   a. additional modules (bashshell)
%   b. organization (split library, pageobjects, tests)
%   c. standard installation using system packages
% 2. improving the testing environment.
%   a. incorporate hubcheck earlier in the development cycle.
%   b. encourage repeated automated testing. once it is setup, its easy to keep doing.
% 3. evangelizing hubcheck to the hubzero team.
%   a. show people how to use hubcheck during their own development.
%   b. encourage people to write their own page objects and tests, and contribute them to the library.
%   c. get people thinking about planning for automated testing while they develop.


\chapter{FUTURE WORK}
\label{chap:future}

The future of HUBcheck is full of opportunities ranging from core library
improvements, to the manifestation of robust test environments, to evangelizing
HUBcheck to the HUBzero Team.

\section{Library Improvements}
\label{sec:future_library}

The HUBcheck library's shell modules are based on the assumption that tests
will be performed in a tool session container over SSH, but having access to a
bash shell on the local machine could open up alternative ways of running
testing currently unavailable to HUBcheck. With local shell access, HUBcheck
could be installed on a hub and run as a user to perform tests without needing
to navigate layers of authentication, reducing false positives stemming from
expired passwords and revoked SSH access. The current bash shell module
template, \xfmodule{hubcheck.bashshell}, builds upon the \xfmodule{pexpect}
module, a Python implementation of Tcl's Expect. The
\xfmodule{hubcheck.bashshell} module aims to implement an interface similar to
that of the \xfmodule{hubcheck.sshshell} module, providing the usual
\xfmethod{send()} and \xfmethod{expect()} methods, but also provides an
\xfmethod{execute()} method described in
\Cref{ssec:hubcheck_shell_modules_interacting_containers}.


The organization of the HUBcheck library could also be improved. Currently, the
core HUBcheck library, HUBzero specific page objects, and tests are all kept in
the same repository and installed together. A better setup would be to install
the core library and allow the HUBzero specific page objects to exist in a
different location. This separation would also make it easier for users to
build and use their own page objects while using the core HUBcheck library as
a frame for developing automation scripts. The tests should be installed in a
separate location, making them easy to update and add to without the need to
reinstall the whole HUBcheck library.

A reorganization of the HUBcheck library should be performed in coordination
with a standardization of the HUBcheck install process. HUBcheck installations
should try to use system libraries whenever possible instead of shipping with
its own copies of libraries to reduce the propagation of security bugs.

\section{Test Environments}
\label{sec:future_testenv}

HUBcheck is one piece of the larger testing puzzle that the HUBzero Team is
trying to solve. The placement of HUBcheck in the development cycle is a bit
off. Right now, it runs on live hubs managed by the HUBzero Team because other
available systems that are used for testing fall short in possessing the
properties of a true testing environment that (1) represents a production
environment, (2) is easily reproducible, and (3) is reliably available.
Developers need a test environment they can quickly launch with a production
compatible hub installation, update with a new feature being developed, and run
HUBcheck tests. This falls in line with the goals of the HUBcheck project of
lowering the barriers to adopting better testing practices earlier in the
development cycle.

\section{Adoption Within the HUBzero Team}
\label{sec:future_adoption}

Another missing piece to the HUBcheck project is an educational component.
Using and setting up HUBcheck to run is undocumented for the most part. Future
work will include evangelizing for HUBcheck, and better testing practices in
general. HUBcheck is not the correct tool for all problems, but fits a unique
niche within the hub environment that discouraged testing in the past and left
hubs in a precarious state. The HUBcheck project helps promote one of the core
phases in the software development process.
