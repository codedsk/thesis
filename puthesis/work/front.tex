%
%  revised  front.tex  2011-09-02  Mark Senn  http://engineering.purdue.edu/~mark
%  created  front.tex  2003-06-02  Mark Senn  http://engineering.purdue.edu/~mark
%
%  This is ``front matter'' for the thesis.
%
%  Regarding ``References'' below:
%      KEY    MEANING
%      PU     ``A Manual for the Preparation of Graduate Theses'',
%             The Graduate School, Purdue University, 1996.
%      TCMOS  The Chicago Manual of Style, Edition 14.
%      WNNCD  Webster's Ninth New Collegiate Dictionary.
%
%  Lines marked with "%%" may need to be changed.
%

  % Dedication page is optional.
  % A name and often a message in tribute to a person or cause.
  % References: PU 15, WNNCD 332.
%%\begin{dedication}
%%  This is the dedication.
%%\end{dedication}

  % Acknowledgements page is optional but most theses include
  % a brief statement of apreciation or recognition of special
  % assistance.
  % Reference: PU 16.
\begin{acknowledgments}
%%  This is the acknowledgments.
% family
% advisor % committee
% MMC and GAH HUBzero Team
% Purdue Perl Mongers
% ECE Grad Office

I would like to express my gratitude to everyone who encouraged and worked with
me through this process of exploring software based automation.

I would like to thank my family for pushing me to always pursue education.

I would like to thank my committee members, Professor Samuel Midkiff, Professor
Mary Comer, Professor Milind Kulkarni, and Professor T. N. Vijaykumar, for
teaching me about research and challenging my scientific thought process.

I would like to thank Dr. Mike McLennan, Mr. George Howlett, and the HUBzero
Team for providing time, resources, and feedback while working on this project.

I would like to thank the members of the Purdue Perl Mongers and GLOSSY for
providing a platform where people can express their software related
experiences and idea.

I would like to thank the staff and faculty of the College of Electrical and
Computer Engineering, especially the members of the Graduate Office who have
helped me start, continue, and finish my graduate education.

\end{acknowledgments}

  % The preface is optional.
  % References: PU 16, TCMOS 1.49, WNNCD 927.
%%\begin{preface}
%%  This is the preface.
%%\end{preface}

  % The Table of Contents is required.
  % The Table of Contents will be automatically created for you
  % using information you supply in
  %     \chapter
  %     \section
  %     \subsection
  %     \subsubsection
  % commands.
  % Reference: PU 16.
\tableofcontents

  % If your thesis has tables, a list of tables is required.
  % The List of Tables will be automatically created for you using
  % information you supply in
  %     \begin{table} ... \end{table}
  % environments.
  % Reference: PU 16.
\listoftables

  % If your thesis has figures, a list of figures is required.
  % The List of Figures will be automatically created for you using
  % information you supply in
  %     \begin{figure} ... \end{figure}
  % environments.
  % Reference: PU 16.
\listoffigures

  % List of Symbols is optional.
  % Reference: PU 17.
%%\begin{symbols}
%%  $m$& mass\cr
%%  $v$& velocity\cr
%%\end{symbols}

  % List of Abbreviations is optional.
  % Reference: PU 17.
%%\begin{abbreviations}
%%  abbr& abbreviation\cr
%%  bcf& billion cubic feet\cr
%%  BMOC& big man on campus\cr
%%\end{abbreviations}

  % Nomenclature is optional.
  % Reference: PU 17.
%%\begin{nomenclature}
%%  Alanine& 2-Aminopropanoic acid\cr
%%  Valine& 2-Amino-3-methylbutanoic acid\cr
%%\end{nomenclature}

  % Glossary is optional
  % Reference: PU 17.
%%\begin{glossary}
%%  chick& female, usually young\cr
%%  dude& male, usually young\cr
%%\end{glossary}

  % Abstract is required.
  % Note that the information for the first paragraph of the output
  % doesn't need to be input here...it is put in automatically from
  % information you supplied earlier using \title, \author, \degree,
  % and \majorprof.
  % Reference: PU 17.
\begin{abstract}

% nobody knows what verification and validation are:
% http://www.easterbrook.ca/steve/2010/11/the-difference-between-verification-and-validation/
% http://www.softwaretestinghelp.com/what-is-verification-and-validation/

The HUBzero Platform is a framework for building websites, referred to as
``hubs,'' that promote research communities through online simulation, data
management, and collaboration. With each software release, the HUBzero Team
dedicates weeks of team members' time toward manually testing, fixing, and
retesting hub components. The unique mixture of environments that make up a hub
makes using existing automated testing solutions hard and shifts the burden of
testing to humans, promoting variation, spot checking of fixes, and other
shortcuts to avoid the high cost of completely retesting the system. With over
twenty hubs being actively managed by the HUBzero Team, manually testing each
one after a software update is resource and time prohibitive.

The HUBcheck library, a collection of Python modules backed by Selenium
WebDriver and Paramiko, was built to help developers write automation scripts
for HUBzero websites and the Debian Linux based virtual containers hosting the
hub's simulation tools. Today, the HUBzero Team is using HUBcheck to perform
automated regression testing on all of its production hubs, regularly testing
areas of the hub that were previously overlooked. In this document, we
investigate how HUBcheck works, introduce three new design patterns that make
writing page object based automation easier, and show how the use of HUBcheck
has helped reduce the number of misconfigured systems during a one year period
of hub upgrades.

\end{abstract}
