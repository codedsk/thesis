%
%  introduction.tex  2012-01-18  Mark Senn  http://engineering.purdue.edu/~mark
%
%  This is the introduction chapter for a simple, example thesis.
%

% A. overview of the hub
%   1. what makes something a hub
%   2. how used and developed
% B. problems
%   1. bugs - related this back to the developement process
%   2. current tools and why they don't solve the problem
% C. Overview of the solution (hubcheck)
% D. Effectiveness of the solution
% E. Overview of the rest of the thesis


\chapter{INTRODUCTION}
\label{chap:introduction}

%\section {What is a hub?}
%\label{sec:what_is_a_hub}

% FIXME:
% tell what a hub component is, hubs are made up of components. which
% components are included in the standard set?

% FIXME:
% what is a hub at a high level?
% hubs like everything else have bugs, hard to find, repeat offenders, good for
% automatic testing, in the rest of the section we will talk about types of
% hubs being targeted, previous attempts to do testing, what out proposed
% solution will be.

% FIXME:
% add wording saying what a hub is, and that we cover the space. what makes
% other web portals not hubs

% outline
% what is a hub
% * open source platform
% * hubs like everything else have bugs, hard to find, repeat offenders
% * good for automatic testing
% * platform with pluggable components
% * different user rolls
% * three features that separate a hub from other websites

HUBzero is an open source software platform for building websites, or hubs,
that support collaborative research, science, and education.  Like all
software, the platform has bugs being introduced and fixed with every release.
Bugs in hubs can be hard to find, and once fixed can be reintroduced in a later
release, which make them good candidates for automated testing. The HUBzero
platform is a dual environment system consisting of a website that uses
pluggable components to provide services to users, and middleware that provides
access to Linux based containers where simulation tools are developed and run.
Due to its complexity, testing on the hub introduces unique challenges
involving the interactions between the web server and middleware that cannot be
addressed by currently available testing software.  The HUBcheck library was
created to address these challenges and provide hub developers with a single
toolkit for performing end user testing on hubs managed by the HUBzero Team at
Purdue University.

\section{Origins of Hub Testing}
\label{sec:problems_with_the_hub}

In 2005, the HUBzero Team supported their first hub, nanoHUB.org. In
2007, the number of hubs in production had risen to four, and by 2012, the team
was supporting twenty five hubs. The release of the open source version allowed
others to begin launching self-supported hubs.  Each deployed hub started from
a single core version of the software and quickly bloomed into its own system
with the standard set of hub components and slightly different configurations
and content.

Despite the best intentions of the HUBzero Team, hub software was, and
still is, being released with bugs. The same economies of scale that allow the
group to rapidly deploy quality software work against the group when combating
erroneous software. A single bug that is discovered on one hub website is
usually also on numerous other hub websites.  In the past, different upgrade
schedules and a lack of upgrade documentation made it hard to tell which hubs
had one of the many bugs reported to the team. Additionally, the lack of a test
suite made it difficult to find bugs before the software was released.

%Different upgrade
%schedules of the hubs and a lack of upgrade documentation make it hard to tell
%which hubs still have one of the known bugs. The lack of a test suite for
%components makes it difficult to find bugs before and after the software is
%released.

Before HUBcheck, the testing of hub components was performed by hand, making it
a time consuming and error prone process.  The manual testing promoted
variation in tests performed. Due to its repetitive nature, each iteration of
testing could be performed a different way, or not performed if forgotten.  The
escalating commitment needed to setup and perform tests encouraged spot
checking of bug fixes instead of a broad retesting of components.


\section{A Hub Specific Testing Solution}
\label{sec:the_hubcheck_solution}

The hub is a controlled environment where the developers understand how the
components are supposed to work. HUBcheck, a set of hub automation libraries,
allows developers to take advantage of this knowledge and perform targeted end
user tests for hub components that the industry standard tools do not support.
HUBcheck provides a collection of tools which can be used to help automate the
testing of how users interact with hub components that would otherwise be
performed by hand.  When compared to testing by hand, HUBcheck reduces testing
time, increases test coverage, and provides a reliable way to reproduce errors.

The HUBcheck libraries focus on two types of automation, hub access through a
secure shell (SSH) and hub access through a web browser. Building on top of
Python's Paramiko libraries and using ideas from Tcl's Expect, HUBcheck
provides functions to easily automate access to different environments through
SSH. This enables developers to test lower level system setup and
configuration, for example, within a tool session container. Similarly,
HUBcheck uses the Selenium library to automate user interactions with the hub
website.  HUBcheck libraries provide abstractions of hub components, that can
be used to write maintainable automation scripts. When combined, these two
automation models can be used to verify that a hub is running as it was
intended.  By using HUBcheck to write test cases for hub components, a hub can
be verified in under a day. The decrease in testing time and increase in tested
components encourages the adoption of automated testing earlier in the
development cycle, where errors cost less to fix.

This thesis describes the inner workings of HUBcheck and how it is utilized by
the HUBzero Team to track and reduce the number of software defects in the
HUBzero Platform. The thesis starts by reviewing related work in
\Cref{chap:related_work} and providing background information about how the hub
works in \Cref{chap:hubzero}.  \Cref{chap:problem} discusses the types of
problems commonly seen on the hub and the reasons they can be hard to test.  In
\Cref{chap:external_libraries}, we look into the external libraries that
HUBcheck depends on to enable developers to automate processes.
\Cref{chap:hubcheck} dives into HUBcheck's modules, explaining how it builds
upon tools like Selenium WebDriver and Paramiko to provide a library for hub
automation and testing. \Cref{chap:page_objects} continues the HUBcheck
discussion by introducing best practices for building maintainable test cases.
Lastly, empirical data from using HUBcheck on hubs managed by the HUBzero Team
is presented in \Cref{chap:solution} and future work for the HUBcheck project
is laid out in \Cref{chap:future}.


% the proof:
%
%
% 3 common examples of regressions
% -------------------------------
% 1. Contribtool - installtool with incorrect permissions
% 2. Known Bad Links - discovered in 2011 vm, some still around in 2012 vm
% 3. packages inside workspace

%the connections between compilers and testing.
%compilers parse code, allow you to identify things that look wierd, could be bugs.
%
%compilers instrument code, testing must also instrument code. in web
%application testing, code instrumentation is being able to identify page
%elements uniquely and reliably by some identifier.

% http://www.youtube.com/watch?v=Wzj8UXdcElo&feature=plcp
% dependency injection
% must treat tests as code.
% framework must have unit tests.

